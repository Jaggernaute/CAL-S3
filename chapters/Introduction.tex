\chapternotnumbered{Introduction} \label{ch:Introduction}


La calculabilit\'e est un domaine fondamental de l'informatique th\'eorique qui explore les limites de ce que les machines peuvent ou ne peuvent pas calculer. Ce document est une compilation compl\`ete de notes et de concepts pour le cours intitul\'e \emph{Calculabilit\'e}, dispens\'e dans le cadre de la Licence 2 mention Informatique. Ce cours vise \`a fournir aux \'etudiants une compr\'ehension approfondie des concepts cl\'es de la calculabilit\'e et de la complexit\'e, en les dotant des comp\'etences n\'ecessaires pour aborder des probl\`emes fondamentaux de l'informatique th\'eorique.

Le cours aborde tout d'abord les concepts fondamentaux de la \textbf{calculabilit\'e \Cref{chap:introduction}}, y compris les th\'eor\`emes de Cantor, le probl\`eme de l'arr\^et, la r\'eduction calculatoire et le th\'eor\`eme de Rice. Ces notions permettent aux \'etudiants de comprendre les limites intrins\`eques des algorithmes et de reconna\^itre les probl\`emes qui ne peuvent \^etre r\'esolus de mani\`ere algorithmique.

Ensuite, le cours explore la \textbf{s\'emantique des langages de programmation \Cref{chap:semantique}}, en s'attardant sur la syntaxe et la s\'emantique du langage WHILE, la d\'efinition d'un interpr\'eteur et la r\'eduction \`a un sous-langage, le langage FOR. Les \'etudiants apprennent \`a d\'efinir formellement les comportements des programmes et \`a explorer les implications de la th\'ese de Church.

Enfin, une attention particuli\`ere est accord\'ee \`a la \textbf{complexit\'e des algorithmes \Cref{chap:complexite}}, o\`u les concepts de co\^ut asymptotique, de classes de complexit\'e P et NP, et le th\'eor\`eme de Cook sont introduits. Ce module permet aux \'etudiants de mieux comprendre l'efficacit\'e des algorithmes et les difficult\'es li\'ees \`a la r\'esolution de probl\`emes NP-complets.
