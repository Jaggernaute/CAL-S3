\chapter{Travaux dirig\'es et travaux pratiques}\label{chap:td_tp}

    \section{Travaux dirig\'es}\label{sec:exos}
    \subsection{TD 1 - Nombres remarquables et ensembles de Cantor}\label{td_1}
    \hspace*{-2.65cm}\textbf{Exercice 1 - Dénombrabilité de \(\mathbb{Z}\) et \(\mathbb{Q}\)}\\
    Nous avons vu en cours que l'ensemble des rationnels positifs, \(\mathbb{Q^+}\) était en bijection avec l'ensemble des entiers naturels \(\mathbb{N}\). On dit alors que l'ensemble des nombres rationnels positifs est dénombrable. La cardinalité de l'ensemble des nombres naturels \(\mathbb{N}\) se note \(\aleph_0\) et on écrit \(\|\mathbb{Q^+}\|=\|\mathbb{N}\|=\aleph_0\).\par

    
    \hspace*{-1.5em}\textbf{1.} Montrer qu'il existe toujours une injection d'une union de deux ensembles dénombrables vers \(\mathbb{N}\). En déduire qu'une union de deux ensembles dénombrables est dénombrable. Citer le théorème utilisé.\\
    \headrule
    Soient \( A \) et \( B \) deux ensembles dénombrables.\\
    Par d\'efinition, il existe des bijections :  
    \begin{equation}
        f: A \to \mathbb{N} \quad \text{et} \quad g: B \to \mathbb{N}
    \end{equation}
    On d\'efinit une fonction injective de \( A \cup B \) vers \( \mathbb{N} \) en intercalant des \'el\'ements de \( A \) et \( B \) :
    \begin{equation}
        h(x) =
        \begin{cases}
            2f(x), & \text{si } x \in A \setminus B\\
            2g(x) + 1, & \text{si } x \in B
        \end{cases}
    \end{equation}
    Cette fonction est injective, car les images de \( A \) sont des nombres pairs et celles de \( B \) sont des nombres impairs. Ainsi, chaque \'el\'ement de \( A \cup B \) a une image distincte dans \( \mathbb{N} \).
    Réciproquement \(f^{-1}\) définit une injection de \(A\cup B\) dans \(\mathbb{N}\).

    \begin{theorem}\label{cantor-schroder-bernstein}
        (Cantor-Schröder-Bernstein, 1895).\\
        Soit E et F deux ensembles. S’il existe une injection de E dans F et une injection de F dans E, E et F sont en bijection.
    \end{theorem}

    \hspace*{-1.5em}\textbf{2.} En déduire que l'ensemble des entiers relatifs \(\mathbb{Z}\) et l'ensemble des rationnels \(\mathbb{Q}\) sont dénombrables :\\
    \(\|\mathbb{Z}\|=\|\mathbb{Q}\|=\|\mathbb{N}\|=\aleph_0\)\par
    \headrule

    \begin{definition}
        \(\mathbb{Z}\).\\
        \(\mathbb{Z} = \mathbb{N}\cup\{-n\colon n\in\mathbb{N}\}\)
    \end{definition}

    Il existe une bijection entre \(\{-n\colon n\in\mathbb{N}\}\) et \(\mathbb{N}\) :
    \begin{equation}
    \begin{split}
        f&:\{-n\colon n\in\mathbb{N}\}\rightarrow\mathbb{N}\\
        f&(x) = -x
    \end{split}
    \end{equation}
    L'ensemble \(\{-n\colon n\in\mathbb{N}\}\) est donc dénombrable. Donc, par application directe du \Cref{cantor-schroder-bernstein}, \(\mathbb{N}\cup\{-n\colon n\in\mathbb{N}\}\) est dénombrable, cela revient à dire que \(\mathbb{Z}\) est dénombrable.\par

    De même, \(\Q=\Q^+\cup\{-q\colon q\in\Q^+\}\). On suppose \(\Q^+\) dénombrable (vu en cours). Il existe une bijection entre \(\Q^+\) et \(\mathbb{N}\) :
    \begin{flalign}
        f&:\Q^+\rightarrow\mathbb{N}\\
        f&(x) = -x
    \end{flalign}
    Donc \(\{-q\colon q\in\Q^+\}\) est aussi dénombrable. En effet :
    \begin{flalign}
        g&\colon\{-q\colon q\in\Q^+\}\rightarrow\N\\
        g&(x)=f(-x)
    \end{flalign}
    est une bijection.\\
    L'ensemble \(\{-q\colon q\in\N\}\) est donc dénombrable. Donc, par application directe du \Cref{cantor-schroder-bernstein}, \(\Q^+\cup\{-q\colon q\in\Q\}\) est dénombrable, cela revient à dire que \(\Q\) est dénombrable.\par

    \hspace*{-2.65cm}\textbf{Exercice 2 - Indénombrabilité de \(\R\)}\\
    \hspace*{-1.5em}\textbf{1.} En raisonnant par l'absurde, montrer que l'ensemble des nombres réels \(\left]0.1\right[\) n'est pas dénombrable. Penser au schéma vu en cours et à la décomposition de tout réel sous forme décimale, \(a=0,a_1a_2a_3\dots\) :
    \[
    \begin{array}{c|ccccccc}
        a_0 &0, &a_{00} &a_{01} &a_{02} &\cdots & a_{0j} & \cdots\\
        \vdots& & & \ddots & & \ddots & & \\
        a_i & 0, &a_{i0} & a_{i1} & a_{i2} & \cdots & a_{ij} & \cdots\\
        \vdots& & & \ddots & & \ddots & & \\
        a_j & 0, &a_{j0} & a_{j1} & a_{j2} & \cdots & a_{jj} & \cdots\\
        \vdots& & & \ddots & & \ddots & & \\
    \end{array}
    \]
    \headrule
    \emph{Considérons} un réel \(b=0,b_1b_2\dots b_j\dots\) tel que pour tout \(j\in\N \text{ par } b_j\neq a_{jj}\). Alors, par définition, \(b\neq a_{j}\) pour tout \(j\in\N\). C'est impossible, car les \(a_j\) sont censés énumérer tous les nombres réels. Donc l'ensemble \(]0;1[\) est indénombrable.

    \hspace*{-1.5em}\textbf{2.} Montrer que les intervalles \(]0;1[\text{ et }]-\frac{\pi}{2};\frac{\pi}{2}[\) ont même cardinal : \(\|]0;1[\|=\|]-\frac{\pi}{2};\frac{\pi}{2}[\|\).\par
    \headrule
    
    \emph{Considérons} les fonctions :
    \begin{equation}
    \begin{split}
        f&\colon\left]-\frac{\pi}{2};\frac{\pi}{2}\right[\rightarrow\left]0;\pi\right[\\
        f&(x)=x+\frac{\pi}{2}
    \end{split}
    \end{equation}
    \begin{equation}
    \begin{split}
        g&\colon\left]0;\pi\right[\rightarrow\left]0;1\right[\\
        g&(x)=\frac{x}{\pi}
    \end{split}
    \end{equation}
    Les fonctions \(f\text{ et }g\) sont bijectives (pour le prouver, on peut regarder leurs fonctions inverses), donc \(f\circ g\) est bijective entre \(]0;1[\text{ et }]-\frac{\pi}{2};\frac{\pi}{2}[\). Ainsi \(]0;1[\text{ et }]-\frac{\pi}{2};\frac{\pi}{2}[\) ont même cardinal.

    \hspace*{-1.5em}\textbf{3.} En utilisant les fonctions bijectives bien connues de \(\left]-\frac{\pi}{2};\frac{\pi}{2}\right[\) dans \(\R\), montrer que \(\R\) est indénombrable.\par
    \headrule
    Il suffit de prendre la fonction \(\tan\). On a démontré dans la fonction précédente que l'on avait montré une bijection entre \(]0;1[\) et \(\R\) et ces ensembles ont même cardinalité. Comme on a montré à la question \(1\) que \(]0;1[\) est indénombrable, \(\R\) l'est aussi.

    \hspace*{-2.65cm}\textbf{Exercice 3 - Cardinalité de \(\R\)}\\
    \begin{Note}
        Le sujet que j'ai récupéré ne corresponds pas au sujet présent sur \emph{Moodle} après la rédaction de ce TD. Certaines questions ont donc légèrement changé pour les exercices 3 et 4.
    \end{Note}
    On admet que tout nombre réel \(a\in\R\) est tel que \(a=sup\left\{q\in\Q\colon q<a\right\}\) (\(supE\) est la borne supérieure de l'ensemble non vide \(E\).\par

    \hspace*{-1.5em}\textbf{1.} Montrer qu'il existe une injection de \(\R\) dans \(\mathcal{P}(\N)\) (ce qui s'écrit avec les notations vues en cours \(\|\R\|\leq\|\mathcal{P}(\N)\|\))\par
    \headrule
    Considérons la fonction :
    \begin{equation}
    \begin{split}
        f&\colon\R\rightarrow\mathcal{P}(\Q)\\
        f&(x)=\left\{q\in\Q\colon q<x\right\}
    \end{split}
    \end{equation}
    Cette fonction est une injection si :
    \begin{equation}
        f(x)=f(x')
    \end{equation}
    Alors : 
    \begin{equation}
        \left\{q\in\Q\colon q<x\right\}=\left\{q\in\Q\colon q<x'\right\}
    \end{equation}
    Donc :
    \begin{equation}
        sup\left\{q\in\Q\colon q<x\right\}=sup\left\{q\in\Q\colon q<x'\right\}
    \end{equation}
    c'est-à-dire \(x=x'\) d'après la propriété admise de \(\R\). On a ainsi montré qu'il existe une injection \(f\) de \(\R\) dans \(\mathcal{P}(\Q)\). On a montré dans l'\textbf{Exercice 1} que \(\Q\) était dénombrable. Il existe une bijection \(g_0\) de \(\Q\) vers \(\N\). Donc, il existe une bijection :
    \begin{equation}
    \begin{split}
        g&\colon\mathcal{P}(\Q)\rightarrow\mathcal{P}(\N)\\
        g&(X)=\left\{g_0(x)\colon x\in X\right\}
    \end{split}
    \end{equation}
    Ainsi, la fonction \(f\circ g\) est une injection de \(\R\) sur \(\mathcal{P}(\N)\).\par

    \hspace*{-1.5em}\textbf{2.} Montrer qu'il existe une bijection de \(\mathcal{P}(\N)\) dans l'ensemble des suites numériques de 0 et 2.\par
    \headrule
    Considérons la fonction :
    \begin{equation}
    \begin{split}
        f&\colon\mathcal{P}(\N)\rightarrow\{0,2\}^\N\\
        f&(x) = (u_n)_{n\in\N}
    \end{split}
    \end{equation}
    \begin{equation}
        \text{avec }u_n=\begin{cases}
            2,&\text{ si }n\in x\\
            0,&\text{ si }n\notin x\\
        \end{cases}
    \end{equation}

    La fonction \(f\) est bien une bijection (sa fonction inverse est triviale).

    \hspace*{-1.5em}\textbf{3.} L'ensemble de \emph{Cantor} \(\mathsf{C}\) est défini comme suit :
    \begin{equation}
        \mathsf{C}=\left\{\sum_{n=0}^\infty\frac{u_n}{3^{n+1}}\colon u_n\in\{0,2\}\right\}
    \end{equation}

    Il s'obtient en soustrayant a l'intervalle ferme \([0;1]\) les intervalles ouverts \(]\frac{1}{3};\frac{2}{3}[\), \(]\frac{1}{9};\frac{2}{2}[\), \(]\frac{7}{9};\frac{8}{9}[,\)\textit{etc}.

    \begin{figure}[H]
        \centering
        \resizebox {.5\textwidth} {!} {
        \begin{tikzpicture}
        \foreach \order in {0,...,4}
            \draw[yshift=-\order*10pt]  l-system[l-system={cantor set, axiom=F, order=\order, step=100pt/(3^\order)}];
        \end{tikzpicture}
        }
        \caption{Quatre premières itérations de la construction de l'ensemble de Cantor \(\mathsf{C}\)}
        \label{fig:Cantor-set}
    \end{figure}
    Montrer qu'il existe une bijection de l'ensemble des suites numériques de 0 et 2 dans\(\mathsf{C}\).\par
    \headrule
    \begin{equation}
        \mathsf{C}:=\left\{\sum_{n=0}^\infty\frac{u_n}{3^{n+1}},(u_n)\in\{0,2\}^\N\right\}
    \end{equation}

    \hspace*{-1.5em}\textbf{4.} Déduire des deux questions précédentes qu'il existe une bijection entre \(\mathsf{C}\) et \(\mathcal{P}(\N)\) (ce qui s'écrit avec les notations du cours \(\|\mathsf{C}\|=\|\mathcal{P}(\N)\|\)) et qu'il existe une injection de \(\mathcal{P}(\N)\) dans \(\R\)\par
    \headrule
    La fonction \(g\circ f\) est une bijection entre \(\mathcal{P}(\N)\) et \(\mathsf{C}\). Par ailleurs, \(\mathsf{C}\subseteq\R\). Par conséquent, \(g\circ f\) est aussi une injection de \(\mathcal{P}(\N)\) dans \(\R\).\par

    \hspace*{-1.5em}\textbf{5.} En déduire que \(\|\R\|=\|\mathcal{P}(\N)\|\). Citer le théorème utilisé.\par
    \headrule
    On a montré à la Question 1 qu'il existait une injection de \(\R\) dans \(\mathcal{P}(\N)\) et montre aux Questions 2, 3 et 4 qu'il existait une injection de \(\mathcal{P}(\N)\) dans \(\R\). Par conséquent, d'après le \Cref{cantor-schroder-bernstein}, il existe bien une bijection de \(\R\) dans \(\mathcal{P}(\N)\), c'est-a-dire \(\|\R\|=\|\mathcal{P}(\N)\|\).\par

    Cantor conjectura que tout ensemble de réels est soit au plus dénombrable, soit à la cardinalité de l'ensemble des réels (\emph{i.e.} du continu). Exprime dans la théorie des ensembles de Zermelo et Fraenkel, cette conjecture s'appelle l'\emph{Hypothèse du Continu} et s'écrit\(\aleph_1=2^{\aleph_0}\) o\`u \(\aleph_1\) est le cardinal successeur de \(\aleph_0\) puisque \(\|\R\|=\|\mathcal{P}(\N)\|=2^{\aleph_0}\)\par

    \hspace*{-2.65cm}\textbf{Exercice 4 - Cardinalité de \(\C\)}\\
    \hspace*{-1.5em}\textbf{1.} Montrer qu'il existe une injection de \(]0;1[^2\) dans \(]0;1[\). on utilisera pour cela la représentation décimale des nombres réels (en choisissant la représentation la plus grande lexicographiquement d'un nombre réel quand elle existe, comme pour 0,5000 a la place de 0,999...). Il faudra définir un nombre réel \(c\) à partir de deux nombres réels \(a=0,a_1a_2a_3\dots\) et \(b=a=0,a_1a_2a_3\dots\).\par
    \headrule
    Considérons la fonction :
    \begin{equation}
    \begin{split}
        f&\colon]0;1[^2\rightarrow]0;1[\\
        \text{si :}&\\
        &a=0,a_1a_2a_3\dots \text{ et } b=0,b_1b_2b_3\dots\\
        \text{alors :}&\\
        f&(a,b) := 
        \begin{cases} 
        a_{\frac{n+1}{2}}, & \text{si } n \text{ est impair} \\
        b_{\frac{n}{2}}, & \text{si } n \text{ est pair}
        \end{cases}
        = 0,a_1b_1a_2b_2\dots
    \end{split}
    \end{equation}
    La fonction \(f\) est une injection si :
    \begin{equation}
        0,a_1b_1a_2b_2\dots = 0,a_1'b_1'a_2'b_2'\dots
    \end{equation}
    alors :
    \begin{equation}
        0,a_1a_2a_3\dots=0,a_1'a_2'a_3'\dots\textbf{ et }0,b_1b_2b_3\dots=0,b_1'b_2'b_3'\dots
    \end{equation}

    \hspace*{-1.5em}\textbf{2.} En déduire que \(\|]0;1[^2\|=\|]0;1[\|\). Citer le théorème utilise.\par
    \headrule
    On vient de définir \(f\) une injection de \(]0;1[^2\text{ dans }]0;1[\). Réciproquement, on peut définir une injection :
    \begin{equation}
    \begin{split}
        g&\colon]0;1[\rightarrow]0;1[^2\\
        g&(x)=(x,x)
    \end{split}
    \end{equation}
    Donc d'après le \Cref{cantor-schroder-bernstein}, il existe une bijection entre \(]0;1[\text{ et }]0;1[^2\).

    \hspace*{-1.5em}\textbf{3.} En se servant des résultats de l'\textbf{Exercice 2}, montrer que \(\|\R\|^2=\|\R\|\). En déduire que les nombres complexes et les nombres réels ont la même cardinalité : \(\|\C\|=\|\R\|\).\par
    \headrule
    D'après l'\textbf{Exercice 2}, il existe une bijection \(f\) de \(\R\) dans \(]0;1[\). D'après la question précédente, il existe une bijection \(g\) de \(]0;1[^2\text{ dans }]0;1[\). Par conséquent, la fonction :
    \begin{equation}
    \begin{split}
        h&\colon\R^2\rightarrow\R\\
        h&=f^{-1}\circ g(f(x),f(y))
    \end{split}
    \end{equation}
    est une bijection. Donc \(\|\R\|^2=\|\R\|\).\par
    \vspace{\baselineskip}
    La fonction de \(\C\) dans \(\R^2\) qui associe à tout nombre complexe la paire de nombre réels constitue de sa partie réelle et imaginaire est une bijection. Donc \(\|\C\|=\|\R^2\|\). Donc si \(\|\R\|^2=\|\R\|\) donc Donc \(\|\C\|=\|\R\|\).
    
    \section{\'Exercices pratiques}\label{sec:etudes_cas}