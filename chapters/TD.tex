\chapter{Travaux dirig\'es et travaux pratiques}\label{chap:td_tp}

    \section{Travaux dirig\'es}\label{sec:exos}
    \subsection{TD 1 - Nombres remarquables et ensembles de Cantor}\label{td_1}
    \hspace*{-2.65cm}\textbf{Exercice 1 - Dénombrabilité de \(\mathbb{Z}\) et \(\mathbb{Q}\)}\\
    Nous avons vu en cours que l'ensemble des rationnels positifs, \(\mathbb{Q^+}\) était en bijection avec l'ensemble des entiers naturels \(\mathbb{N}\). On dit alors que l'ensemble des nombres rationnels positifs est dénombrable. La cardinalité de l'ensemble des nombres naturels \(\mathbb{N}\) se note \(\aleph_0\) et on écrit \(\|\mathbb{Q^+}\|=\|\mathbb{N}\|=\aleph_0\).\par

    
    \hspace*{-1.5em}\textbf{1.} Montrer qu'il existe toujours une injection d'une union de deux ensembles dénombrables vers \(\mathbb{N}\). En déduire qu'une union de deux ensembles dénombrables est dénombrable. Citer le théorème utilisé.\\
    \noindent\makebox[\linewidth]{\rule{\linewidth}{0.4pt}}
    Soient \( A \) et \( B \) deux ensembles dénombrables.\\
    Par d\'efinition, il existe des bijections :  
    \begin{equation}
        f: A \to \mathbb{N} \quad \text{et} \quad g: B \to \mathbb{N}
    \end{equation}
    On d\'efinit une fonction injective de \( A \cup B \) vers \( \mathbb{N} \) en intercalant des \'el\'ements de \( A \) et \( B \) :
    \begin{equation}
        h(x) =
        \begin{cases}
            2f(x), & \text{si } x \in A \setminus B\\
            2g(x) + 1, & \text{si } x \in B
        \end{cases}
    \end{equation}
    Cette fonction est injective, car les images de \( A \) sont des nombres pairs et celles de \( B \) sont des nombres impairs. Ainsi, chaque \'el\'ement de \( A \cup B \) a une image distincte dans \( \mathbb{N} \).
    Réciproquement \(f^{-1}\) définit une injection de \(A\cup B\) dans \(\mathbb{N}\).

    \begin{theorem}\label{cantot-schroder-bernstein}
        (Cantor-Schröder-Bernstein, 1895).\\
        Soit E et F deux ensembles. S’il existe une injection de E dans F et une injection de F dans E, E et F sont en bijection.
    \end{theorem}

    \hspace*{-1.5em}\textbf{2.} En déduire que l'ensemble des entiers relatifs \(\mathbb{Z}\) et l'ensemble des rationnels \(\mathbb{Q}\) sont dénombrables :\\
    \(\|\mathbb{Z}\|=\|\mathbb{Q}\|=\|\mathbb{N}\|=\aleph_0\)\\
    \noindent\makebox[\linewidth]{\rule{\linewidth}{0.4pt}}

    \begin{definition}
        \(\mathbb{Z}\).\\
        \(\mathbb{Z} = \mathbb{N}\cup\{-n\colon n\in\mathbb{N}\}\)
    \end{definition}

    Il existe une bijection entre \(\{-n\colon n\in\mathbb{N}\}\) et \(\mathbb{N}\) :
    \begin{flalign}
        f&:\{-n\colon n\in\mathbb{N}\}\rightarrow\mathbb{N}\\
        f&(x) = -x
    \end{flalign}
    L'ensemble \(\{-n\colon n\in\mathbb{N}\}\) est donc dénombrable. Donc, par application directe du \Cref{cantot-schroder-bernstein}, \(\mathbb{N}\cup\{-n\colon n\in\mathbb{N}\}\) est dénombrable, cela revient à dire que \(\mathbb{Z}\) est dénombrable.\par

    De même, \(\Q=\Q^+\cup\{-q\colon q\in\Q^+\}\). On suppose \(\Q^+\) dénombrable (vu en cours). Il existe une bijection entre \(\Q^+\) et \(\mathbb{N}\) :
    \begin{flalign}
        f&:\Q^+\rightarrow\mathbb{N}\\
        f&(x) = -x
    \end{flalign}
    Donc \(\{-q\colon q\in\Q^+\}\) est aussi dénombrable. En effet :
    \begin{flalign}
        g&\colon\{-q\colon q\in\Q^+\}\rightarrow\N\\
        g&(x)=f(-x)
    \end{flalign}
    est une bijection.\\
    L'ensemble \(\{-q\colon q\in\N\}\) est donc dénombrable. Donc, par application directe du \Cref{cantot-schroder-bernstein}, \(\Q^+\cup\{-q\colon q\in\Q\}\) est dénombrable, cela revient à dire que \(\Q\) est dénombrable.\par

    \hspace*{-2.65cm}\textbf{Exercice 2 - Indénombrabilité de \(\R\)}\\
    \hspace*{-1.5em}\textbf{1.} En raisonnant par l'absurde, montrer que l'ensemble des nombres réels \(\left]0.1\right[\) n'est pas dénombrable. Penser au schéma vu en cours et à la décomposition de tout réel sous forme décimale, \(a=0,a_1a_2a_3\dots\) :
    \[
    \begin{array}{c|ccccccc}
        a_0 &0, &a_{00} &a_{01} &a_{02} &\cdots & a_{0j} & \cdots\\
        \vdots& & & \ddots & & \ddots & & \\
        a_i & 0, &a_{i0} & a_{i1} & a_{i2} & \cdots & a_{ij} & \cdots\\
        \vdots& & & \ddots & & \ddots & & \\
        a_j & 0, &a_{j0} & a_{j1} & a_{j2} & \cdots & a_{jj} & \cdots\\
        \vdots& & & \ddots & & \ddots & & \\
    \end{array}
    \]
    \noindent\makebox[\linewidth]{\rule{\linewidth}{0.4pt}}

    \emph{Considérons} un réel \(b=0,b_1b_2\dots b_j\dots\) tel que pour tout \(j\in\N \text{ par } b_j\neq a_{jj}\). Alors, par définition, \(b\neq a_{j}\) pour tout \(j\in\N\). C'est impossible, car les \(a_j\) sont censés énumérer tous les nombres réels. Donc l'ensemble \(]0;1[\) est indénombrable.

    \hspace*{-1.5em}\textbf{2.} Montrer que les intervalles \(]0;1[\text{ et }]-\frac{\pi}{2};\frac{\pi}{2}[\) ont même cardinal : \(\|]0;1[\|=\|]-\frac{\pi}{2};\frac{\pi}{2}[\|\).
    \noindent\makebox[\linewidth]{\rule{\linewidth}{0.4pt}}
    \emph{Considérons} les fonctions :
    \begin{flalign}
        f&\colon\left]-\frac{\pi}{2};\frac{\pi}{2}\right[\rightarrow\left]0;\pi\right[\\
        f&(x)=x+\frac{\pi}{2}
    \end{flalign}
    \begin{flalign}
        g&\colon\left]0;\pi\right[\rightarrow\left]0;1\right[\\
        g&(x)=\frac{x}{\pi}
    \end{flalign}
    Les fonctions \(f\text{ et }g\) sont bijectives (pour le prouver, on peut regarder leurs fonctions inverses), donc \(f\circ g\) est bijective entre \(]0;1[\text{ et }]-\frac{\pi}{2};\frac{\pi}{2}[\). Ainsi \(]0;1[\text{ et }]-\frac{\pi}{2};\frac{\pi}{2}[\) ont même cardinal.

    \hspace*{-1.5em}\textbf{3.} En utilisant les fonctions bijectives bien connues de \(\left]-\frac{\pi}{2};\frac{\pi}{2}\right[\) dans \(\R\), montrer que \(\R\) est indénombrable.\\
    \noindent\makebox[\linewidth]{\rule{\linewidth}{0.4pt}}
    Il suffit de prendre la fonction \(\tan\). On a démontré dans la fonction précédente que l'on avait montré une bijection entre \(]0;1[\) et \(\R\) et ces ensembles ont même cardinalité. Comme on a montré à la question \(1\) que \(]0;1[\) est indénombrable, \(\R\) l'est aussi.

    \hspace*{-2.65cm}\textbf{Exercice 3 - Cardinalité de \(\R\)}\\
    
    
    \section{\'Exercices pratiques}\label{sec:etudes_cas}