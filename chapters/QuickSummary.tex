% \setcounter{chapter}{16} % "Q" (17th letter, but latex first increments -> 17-1=16)
% \chapter{Résumé Rapide}
\chapternotnumbered{Résumé Rapide} \label{ch:Quick Summary}

\vspace{2ex} % espace vertical supplémentaire, car la lettre Q a une longue queue

\textbf{Calculabilit\'e \Cref{chap:introduction}} : Traite des th\'eor\`emes de Cantor, du probl\`eme de l'arr\^et, de la r\'eduction calculatoire et du th\'eor\`eme de Rice.

\textbf{S\'emantique des langages de programmation \Cref{chap:semantique}} : Analyse la syntaxe et la s\'emantique du langage WHILE, la mise en \oe uvre d'un interpr\'eteur, et l'\'etude du langage FOR.

\textbf{Complexit\'e des algorithmes \Cref{chap:complexite}} : Explique les concepts de complexit\'e asymptotique, les classes de complexit\'e P et NP, et le th\'eor\`eme de Cook.

