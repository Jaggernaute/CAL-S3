\chapter{Introduction \`a la Calculabilit\'e}\label{chap:introduction}
    La \emph{calculabilit\'e} est un domaine de la logique mathématique et de l'informatique théorique qui vise à identifier les limites de ce qui peut être calculé par un algorithme. Cette discipline ne se préoccupe pas de l'efficacité des algorithmes, qui est l'objet de la théorie de la complexité, mais seulement de l'existence ou de la non-existence d'un algorithme qui résout un problème donné, sur un ordinateur idéalisé tel qu'une machine de Turing, démontrée équivalente en possibilités à tous les ordinateurs existants. La notion la plus centrale de la calculabilité est celle de fonction calculable. Son intérêt est justifié par la \emph{thèse de Church \autocite{church_unsolvable_1936}}, qui affirme que les fonctions calculables avec la définition formelle correspondent exactement aux fonctions qui peuvent être calculées par un humain ou une machine, dans le monde physique. Cependant, la calculabilité ne se limite pas à séparer les fonctions calculables des fonctions non calculables, mais cherche aussi à comparer les fonctions non calculables entre elles, en s'appuyant sur la notion de réduction pour affirmer (informellement) que certaines fonctions sont plus "incalculables" que d'autres. Les niveaux d'incalculabilité sont appelés \emph{degrés de Turing}, et une partie de la calculabilité consiste à étudier leur structure. 
    
    \section{Notions fondamentales de la calculabilit\'e}\label{sec:notions_fondamentales}
    \subsection{Programmes}
    \begin{definition}
        Programme :\\
        Un programme est un texte \emph{fini}.\\
        On notera \(\mathbb{P}\) l'ensemble des programmes. C'est un ensemble fini.\\
        \begin{Note}
            Précision :\\
            Ici aucune hypoth\`ese n'est faite sur le jeu de charact\`eres, le lexique ou le langage de programmation.
        \end{Note}
    \end{definition}
    \subsection{Fonctions}
    \begin{definition}
        Fonction :\\
        Soient \(\mathbb{D}\) et \(\mathbb{D'}\) deux ensembles de valeurs, une \emph{fonctions} \(f:\mathbb{D}\rightarrow\mathbb{D'}\) est un ensemble de paires \(\langle v,v'\rangle\) telles que \(v\in\mathbb{D}\) et \(v'\in\mathbb{D'}\), et vérifie :
        \begin{equation}
            \forall v\in\mathbb{D}, \forall v',v''\in\mathbb{D'}, \left[\langle v,v'\rangle\in f\wedge\langle v,v''\rangle\in f \Rightarrow v'=v''\right]
        \end{equation}
        On appelle \(v\) une \emph{entrée} (en pratique, ce serait un \emph{paramètre}), et \(v'\) une \emph{sortie} ou \emph{l'image de \(v\) par \(f\)}.\\
        \(\mathbb{D}\) est appelé son \emph{domaine} et \(\mathbb{D'}\) son \emph{co-domaine}.\par
        L'ensemble des valeurs \(v\) telles que \(\langle v,v'\rangle\) appartient \`a une fonction \(f\) est le \emph{domaine de d\'efinition} de \(f\colon\{v\in\mathbb{D}|\exists v'.\langle v,v'\rangle\in f\}\).\\
        Il peut \^etre infini.\\
        Si le domaine de d\'efinition de \(f\) est \'egal au domaine \(\mathbb{D}\) on dit que la fonction \(f\) est \emph{totale}. Elle est d\'efinie partout, \(\forall v\in\mathbb{D}.\exists v'.\langle v,v'\rangle\in f\).\\
        Sinon, on dit qu'elle est \emph{partielle}, il existe des points ou elle n'est pas d\'efinie. Si \(x\) est un de ces point, faire r\'ef\'erence \`a \(f(x)\) n'\`a aucun sens. 
    \end{definition}

    Il existe deux points de vue pour décrire une fonction :
    \begin{itemize}
        \item \textbf{Point de vue extensionnel : } On considère qu'une fonction est entièrement définie par son \emph{graphe} (l'ensemble des couples \((x,y)\) qui la constitue) :
        \begin{example}
            \begin{equation}
                d = \{(0,0),(1,2),(2,4),(3,6),\dots\}
            \end{equation}
        \end{example}
        Le point de vue \emph{extensionnel} sur les fonctions les résume à leur comportement \emph{d'entrée-sortie}.

        \item \textbf{Point de vue intensionnel : } Ici, on considère une fonction comme un procédé de calcul défini par un enchaînement d'opérations.
        \begin{example}
            \begin{flalign}
            f&:x\mapsto x+x\\
            g&:x\mapsto2x
            \end{flalign}
            Ici, les définitions des fonctions \(f\) et \(g\) dénotent différentes fonctions dont on pourrait montrer qu'elles sont extensionnellement équivalentes parce qu'elles produisent partout le même effet.\\
            Ce principe est à rapprocher de la définition d'un ensemble en compréhension d'un ensemble :
            \begin{equation}
                d=\{(x,y)|y=x+x\}
            \end{equation}
        \end{example}
    \end{itemize}

    \subsection{Problèmes de decision}
    Certaines \emph{fonctions} sont ce que l'on appelle des \emph{problèmes de décision} (voir \Cref{def:pb-decision}). Elles définissent si une propriété est \emph{vraie} en rendant les valeurs \emph{vrai} ou \emph{faux}.\\
    L'ensemble \(\{vrai,faux\}\) est noté \(\mathbb{B}\).
    \begin{Todo}
        Complément.\\
        Mettre en évidence le lien entre les problèmes de décision et l'algèbre de Boole.
    \end{Todo}
    
    \begin{definition}\label{def:pb-decision}
        Problème de décision :\\
        Un problème de décision est une fonction \(f\) totale de \(\mathbb{D}\) dans \(\mathbb{B}\). On écrit \(f\colon\mathbb{D}\rightarrow\mathbb{B}\).
    \end{definition}

    Une \emph{instance} d'un problème de décision, une donnée de \(\mathbb{D}\) pour laquelle on veut résoudre le problème. On dit qu'une instance est \emph{positive} si son image est \(\mathtt{vrai}\) et \emph{négative} si son image est \(\mathtt{fausse}\). L'ensemble des problèmes de décision est \emph{beaucoup plus gros} que celui des programmes.

    \subsection{S\'emantique}
        \begin{definition}\label{def:sementique}
        Sémantique d'un programme :\\
        La sémantique des programmes est modélisé par une fonction \(\mathcal{SEM}\) définie des programmes \(\mathbb{P}\) vers les fonctions \(\D\rightarrow\D\). On écrit \(\mathcal{SEM}\colon\mathbb{P}\rightarrow\D\rightarrow\D\) et \(\mathcal{SEM}(\texttt{p})\) la sémantique du programme \texttt{p}.
    \end{definition}

    Puisque que la sémantique d'un programme est une fonction, \(\mathcal{SEM}(\texttt{p})\) est une fonction qui demande à être appliquée \`a des données. On écrira donc \(\mathcal{SEM}(\texttt{p})(v)\) pour designer le résultat de l'application de la sémantique d'un programme \texttt{p} a un programme \(v\).
    \begin{Note}Différencier la sémantique du programme.\\
        Il est tentant d'écrire \(\texttt{p}(v)\), mais \texttt{p} n'est qu'un texte, il ne peut rien tout seul.
    \end{Note}
    La sémantique d'un programme est donc un ensemble de paires \(\langle valeur, image\rangle\).\par
    
    Quand l'exécution d'un programme ne répond rien pour certaines valeurs d'entrée, boucle, s'arrête brutalement sur une condition d'erreur (ex. division par 0), la fonction \(\mathcal{SEM}\) n'est pas définie sur ces valeurs.\par
    \begin{Note} Définir le non défini.\\
        On ne peut pas augmenter le co-domaine d'une nouvelle valeur, \texttt{non-def}, qui servirait à représenter qu'une fonction n'est pas définie en un point \(v\), tel que \(\langle v,\texttt{non-def}\rangle\in \mathcal{SEM}(\texttt{p})\) plutôt que \(\lnot\exists w.\langle v,w\rangle\in \mathcal{SEM}(\texttt{p})\).\\
        Cela reviendrait à définir le non défini, cela reviendrait à dire que toutes les anomalies sont prévues, or c'est impossible.
    \end{Note}

    D'autre part, la théorie de la calculabilité montre qu'il n'existe pas de modèle de calcul qui ne réalise que des fonctions totales, et les réalise toutes. On s'en tiendra donc à l'idée que des paires peuvent manquer dans la définition d'une fonction.\par
    Il existe aussi des programmes mal formes, qui seront refusés par un compilateur par exemple, du point de vue de la sémantique un programme mal forme n'est pas un programme.\par

    \subsection{Calculabilité et décidabilité}
    \begin{definition}
        Fonctions calculables.\\
        Lorsqu'une fonction totale est la sémantique d'un programme, on dit qu'elle est \emph{calculable}. On dit sinon qu'elle est \emph{non calculable}.
    \end{definition}
        \begin{definition}
        Problèmes décidables.\\
        Lorsqu'un problème de décision est calculable, on dit qu'il est \emph{décidable}. On dit sinon qu'il est \emph{indécidable} ou non décidable.
    \end{definition}
    \begin{result}
        La fonction qui teste l'arrêt d'un programme sur une donnée, tous deux passés en paramètres, n'est pas calculable. \autocite{turing1936}
    \end{result}
    \begin{definition}
            Pour chaque valeur, \(v\in\D\) l'exécution d'un programme \texttt{p} peut s'arrêter, \(\exists w.\langle v,w\rangle\in\mathcal{SEM}(\texttt{p})\), ou boucler \(\lnot\exists w.\langle v,w\rangle\in\mathcal{SEM}(\texttt{p})\).\\
            Le problème de l'arrêt consiste à réaliser une fonction \(\text{Arr\^et}\colon\mathbb{P}\times\D\rightarrow\mathbb{B}\) telle que \(\texttt{p},v\mapsto\exists w.\langle v,w\rangle\in\mathcal{SEM}(\texttt{p})\)
    \end{definition}
    \begin{result}
        Les fonctions qui prennent des programmes en paramètre et dont le résultat dépend uniquement de leur sémantique ne sont jamais calculables. C'est le théorème de Rice.\autocite{rice1953}
    \end{result}
    \begin{Note} Sur le théorème de Rice.\\
        Le théorème de Rice ne pose de limites que sur l'analyse statique du programme, c'est-à-dire sur tous les problèmes qui peuvent être déduits sans lancer le programme. Il ne rentre donc pas en conflit avec les fonctions de compilation, même si les compilateurs actuels sont de plus en plus puissants et performants, leur analyse du code reste statique.\\
        Lectures conseillées :
        \begin{itemize}
            \item Sémantiques mécanisées, Xavier Leroy, Collège de France
            \item Data flow analysis, Thomas Jensen, Université de Rennes
        \end{itemize}
    \end{Note}

    \subsection{La thèse de Church}
    \begin{definition}
        Modèle de calcul\\
        Un modèle de calcul est la définition formelle d'une règle de calcul qui compose des opérations élémentaires ne recourant chacune qu'à des moyens finis.
    \end{definition}
    \begin{result}
        La notion de calculabilité ne dépend pas du modèle de calcul.
    \end{result}

    De nombreuses propositions de langages de programmation et de sémantiques de programmes ont été faites. À chaque fois, les mêmes capacités de calcul ont été trouvées. Autrement dit, à chaque fois, les mêmes fonctions étaient non calculables. Cette observation se formalise dans une conjecture, appelée la \emph{Thèse de Church} \autocite{church_unsolvable_1936}, énoncée par Alonzo Church et ses collègues à partir de 1936.\\
    On dit des modèles de calcul qui permettent de programmer tout ce qui est calculable qu'ils ont la puissance du calculable, ou qu'ils sont équivalents à une machine de \emph{Turing universelle}, du nom de l'un des premiers modèles de la calculabilité, ou encore qu'ils sont calculatoirement complets. La plupart des langages de programmation ont la puissance du calculable.

    \subsection{Programmes WHILE}
    Les programmes WHILE dont la principale caractéristique est d'offrir une boucle \texttt{while c do i od} qui commande d'exécuter l'instruction \texttt{i} tant que la condition \texttt{c} est vraie. La condition \texttt{c} peut cesser d'être vraie par l'effet de l'exécution de l'instruction \texttt{i}. Puisque l'exécution de ces programmes peut boucler, leur sémantique peut consister en une fonction partielle. En contrepartie, ils ont la puissance du calculable.

    \subsection{Programmes FOR}
    Le modèle des programmes FOR, en remplace la boucle \texttt{while c do i od} par une boucle \texttt{for n do i od} qui commande d'exécuter \texttt{n} fois l'instruction \texttt{i}. Ce nombre d'itérations \texttt{n} est calculé avant d'exécuter la boucle.
    \begin{result}
        Il n'est pas possible de réaliser toutes les fonctions calculables dans un modèle de calcul où l'exécution des programmes termine toujours.
    \end{result}

    \subsection{Fonction d'Ackermann}
    Le moyen le plus direct de montrer qu'il existe des fonctions calculables non programmables dans le langage FOR est d'en exhiber un exemple concret tel que la fonction d'Ackermann :
    \begin{equation}
        A(m, n) = 
      \begin{cases}
         n+1 & \mbox{si } m = 0 \\
         A(m-1, 1) & \mbox{si } m > 0 \mbox{ et } n = 0 \\
         A(m-1, A(m, n-1)) & \mbox{si } m > 0 \mbox{ et } n > 0.
      \end{cases}
    \end{equation}

    \subsection{Complexité des fonctions}
    \begin{result}
         Le coût minimal pour les réaliser par programme est une propriété intrinsèque des fonctions. Dans certains cas, ce coût est énorme et rend la fonction irréalisable en pratique. On utilise d'ailleurs le terme \emph{impraticable}.
    \end{result}
    Pour certaines fonctions, on pourra établir un coût de calcul \emph{minimum intrinsèque}. Pour d'autres, on ne connaît qu'un coût de calcul \emph{minimum empirique} : celui du programme le plus efficace à ce jour qui les réalise.

    \subsection{\(\mathcal{P}\) vs. \(\mathcal{NP}\)}

    \begin{definition}
        Les fonctions \(\mathcal{P}\)\\
        On appelle \(\mathcal{P}\) l'ensemble des fonctions dont le coût du calcul en fonction de la taille des entrées est majoré par une fonction polynomiale de la taille de l'entrée.
    \end{definition}
    \begin{definition}
        Les fonctions \(\mathcal{NP}\)\\
        On appelle \(\mathcal{NP}\) l'ensemble des fonctions dont le coût de la vériffication des réponses est majoré par une fonction polynomiale.
    \end{definition}

    \begin{result}
        Les ensembles P et N P ont été déffinis vers 1970 ; ils s'intègrent dans une classification des fonctions qui est riche, mais des questions restent sans réponse à leur sujet.\\
        On ne sait pas si \(\mathcal{P}\neq\mathcal{NP}\), cela constitue l'un des problèmes ouverts les plus importants de notre époque, en mathématiques et en informatique.
    \end{result}

    \subsection{La machine de Turing}
    Cette machine de papier a été déffinie en 1936 par Alan Turing et a servi de modèle théorique à la création des calculateurs tels que nous les connaissons actuellement.\\
    Les notions aussi fondamentales que celles de l'indécidabilité ou de l'interprétation d'un programme au moyen d'un autre programme ont été déffinies par Alan Turing lui-même, dès 1936. De ce modèle ont découlé les déffinitions d'autres modèles basés sur des automates.
    
  \section{Th\'eor\`emes de Cantor}\label{sec:theoremes_cantor}
  
  
  \section{Le probl\`eme de l'arrêt}\label{sec:probleme_arret}
  
  
  \section{La r\'eduction calculatoire}\label{sec:reduction_calculatoire}
  
  
  \section{Le th\'eor\`eme de Rice}\label{sec:theoreme_rice}