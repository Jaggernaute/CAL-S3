\chapter{Introduction \`a la Calculabilit\'e}\label{chap:introduction}
    \section{Notions fondamentales de la calculabilit\'e}\label{sec:notions_fondamentales}
    \begin{definition}
        Programme :\\
        Un programme est un texte \emph{fini}.\\
        On notera \(\mathbb{P}\) l'ensemble des programmes. C'est un ensemble fini.\\
        \begin{Note}
            Précision :\\
            Ici aucune hypoth\`ese n'est faite sur le jeu de charact\`eres, le lexique ou le langage de programmation.
        \end{Note}
    \end{definition}
    \begin{definition}
        Fonction :\\
        Soient \(\mathbb{D}\) et \(\mathbb{D'}\) deux ensembles de valeurs, une \emph{fonctions} \(f:\mathbb{D}\rightarrow\mathbb{D'}\) est un ensemble de paires \(\langle v,v'\rangle\) telles que \(v\in\mathbb{D}\) et \(v'\in\mathbb{D'}\), et vérifie :
        \begin{equation}
            \forall v\in\mathbb{D}, \forall v',v''\in\mathbb{D'}, \left[\langle v,v'\rangle\in f\wedge\langle v,v''\rangle\in f \Rightarrow v'=v''\right]
        \end{equation}
        On appelle \(v\) une \emph{entrée} (en pratique, ce serait un \emph{paramètre}), et \(v'\) une \emph{sortie} ou \emph{l'image de \(v\) par \(f\)}.\\
        \(\mathbb{D}\) est appelé son \emph{domaine} et \(\mathbb{D'}\) son \emph{co-domaine}.
    \end{definition}
  
  \section{Th\'eor\`emes de Cantor}\label{sec:theoremes_cantor}
  
  
  \section{Le probl\`eme de l'arrêt}\label{sec:probleme_arret}
  
  
  \section{La r\'eduction calculatoire}\label{sec:reduction_calculatoire}
  
  
  \section{Le th\'eor\`eme de Rice}\label{sec:theoreme_rice}